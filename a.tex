选项分析
\begin{enumerate}
	\item {选项A:物体\( A \)可能相对小车向右运动}
	若物体相对小车向右运动,需\( f' = -f_{\text{滑}} \)(向左),则:
	\[
	F_{\text{弹}}' = 8 + f_{\text{滑}} \geq 8 + 5 = 13 \, \text{N}
	\]
	但弹簧拉力由形变量决定,小车加速瞬间弹簧形变量不变,\( F_{\text{弹}}' = 5 \, \text{N} \),无法满足上式。因此A\cuo。
	
	
	\item {选项B:物体\( A \)受到的弹簧拉力可能增大}
	弹簧拉力由形变量决定,小车加速瞬间弹簧未发生新的形变,故\( F_{\text{弹}}' = 5 \, \text{N} \)。因此B\cuo。
	
	
	\item {选项C:物体\( A \)受到的摩擦力大小可能不变}
	已知\( F_{\text{弹}}' = 5 \, \text{N} \),代入受力方程得:
	\[
	f' = 8 - 5 = 3 \, \text{N} \quad (\text{向右})
	\]
	但需考虑摩擦力的方向变化:
	- 初始时摩擦力向左(\( f = 5 \, \text{N} \));
	- 若最大静摩擦力\( f_{\text{max}} \geq 3 \, \text{N} \),则加速后摩擦力可向右,大小为\( 3 \, \text{N} \);
	- 若\( f_{\text{max}} \geq 5 \, \text{N} \),也可能存在**摩擦力大小不变**的情况(例如:摩擦力向左,大小仍为\( 5 \, \text{N} \),此时\( F_{\text{弹}}' + f' = 5 + 5 = 10 \, \text{N} > 8 \, \text{N} \),实际需结合静摩擦力范围调整,存在“大小不变”的可能)。
	
	因此C正确\dui。
	
	
	\item {选项D:物体\( A \)受到的摩擦力一定减小}
	由选项C的分析,摩擦力可能从向左的\( 5 \, \text{N} \)变为向右的\( 3 \, \text{N} \)(减小),但也可能保持\( 5 \, \text{N} \)(大小不变)。因此D\cuo。
\end{enumerate}	
正确选项为\(\boxed{C}\)。