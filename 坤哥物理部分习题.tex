\chapter{斜面}
\begin{center}
	\begin{tikzpicture}[>=Stealth, scale=1, every node/.style={font=\small\itshape}]
		% 图1:m受力分析
		\begin{scope}
			\def\ang{37}
			\coordinate (O) at (0,0);
			% \draw[dashed] (-1,0) -- (3,0); % Horizontal line
			\draw[thick] (0,0) -- (3, {3*tan(\ang)});
			\node at (2.5, 2.3) {斜面};
			
			% Object
			\coordinate (P) at (1.5, {1.5*tan(\ang)});
			\begin{scope}[shift={(P)}, rotate=\ang]
				\draw[thick, fill=white] (-0.35,0) rectangle (0.35,0.4);
				\node at (0,0.2) {$m$};
				\coordinate (Center) at (0, 0.2);
				
				% Forces
				\draw[->, red, thick] (Center) -- (0, 1.2) node[above] {$N_m$}; % Support
				\draw[->, red, thick] (Center) -- (1.0, 0.2) node[right] {$f_m$}; % Friction (up slope)
			\end{scope}
			% Gravity
			\begin{scope}[shift={(P)}]
				\draw[->, blue, thick] (-0.1,{0.2*cos(\ang)}) -- (-0.1, -1.5) node[right] {$mg$};
				\draw[fill=brown] (-0.1,{0.2*cos(\ang)}) circle(0.05);
			\end{scope}
			
			\node[below] at (1.5, -1.5) {图1:物块$m$受力分析};
		\end{scope}
		
		% 图2:M受力分析
		\begin{scope}[shift={(6,0)}]
			\def\ang{37}
			\def\w{3.5}
			\pgfmathsetmacro{\h}{\w*tan(\ang)}
			
			\coordinate (A) at (0,0);
			\coordinate (B) at (\w,0);
			\coordinate (C) at (\w,\h);
			
			\draw[thin] (A) -- (B) -- (C) -- cycle;
			\node at ($(A)!0.6!(B)!0.3!(C)$) {$M$};
			
			% Forces from m on M
			\coordinate (Contact) at ($(A)!0.65!(C)$);
			\begin{scope}[shift={(Contact)}, rotate=\ang]
				% N'_m (Pressure)
				\draw[->, violet,very thick] (0,0) -- (0, -0.8) node[right] {$N'_m$};
				% f'_m (Friction from m, down slope)
				\draw[->, violet, very thick] (0,0) -- (-0.8, 0) node[above left] {$f'_m$};
			\end{scope}
			
			% Gravity M
			\coordinate (MCenter) at (2, 0.8);
			\draw[->, blue, thick] (MCenter) -- ++(0, -1.5) node[right] {$Mg$};
			
			% Ground Support
			\draw[->, red, thick] (2, 0) -- (2, 1.5) node[left,midway] {$N_{\text{地}}$};
			
			% Ground Friction (M pushed right by N'm_x and f'm_x? No. Check x components.)
			% N'm is perp to slope (down-right). +x component.
			% f'm is down slope (down-left). -x component.
			% If N'm_x > f'm_x, M tends right.
			% N'm_x = mg cos theta sin theta.
			% f'm_x = f'm cos theta = m(g sin theta - a) cos theta.
			% If a=g sin theta (smooth), f'm=0. M pushed right by N'm. Friction left.
			% Generally friction f_ground is left.
			\draw[->, red, thick] (2, 0) -- (0.8, 0) node[below] {$f_{\text{地}}$};
			
			\draw[thick] (-0.5,0) -- (\w+0.5,0);
			\node[below] at (2, -1.5) {图2:斜面$M$受力分析};
		\end{scope}
	\end{tikzpicture}
\end{center}

\chapter{牛二}
如图\ref{fig3.1}所示,质量为10 kg的物体\( A \)拴在一个被水平拉伸的弹簧一端,弹簧的拉力为5 N时,物体\( A \)处于静止状态。若小车突然以\( 0.8 m/s^2 \)的加速度向右加速运动,重力加速度\( g = 10 m/s^2 \),则()
\begin{enumerate}[label=\Alph*.\ ]
	\item 物体\( A \)可能会相对小车向右运动
	\item  物体\( A \)受到的弹簧的拉力可能增大
	\item  物体\( A \)受到的摩擦力大小可能不变
	\item  物体\( A \)受到的摩擦力一定减小
\end{enumerate}

\begin{figure}[H]
	\begin{center}
		\begin{tikzpicture}[scale=1.2]
			% 小车
			\draw[fill=gray!20] (0,-0.8)--(4,-0.8)--(4,-0.7)--(4,1)--(3.9,1)--(3.9,-0.7)--(0,-0.7)--cycle;
			\draw[fill] (0.5,-0.9) circle (0.1);
			\draw[fill] (3.5,-0.9) circle (0.1);
			\draw[pattern=north east lines] (0,-1.1) rectangle (4,-1); % 地面
			
			% 物体A
			\draw[fill=gray!30,thick] (1,-0.7) rectangle (2,0.3);
			\node at (1.5,-.2) {$A$};
			
			% 弹簧coil也可以
			\draw[decorate,decoration={zigzag, segment length=4pt, amplitude=4pt}] (2,-0.2) -- (3.9,-0.2);
			
		\end{tikzpicture}
	\end{center}
	\caption{习题}
	\label{fig3.1}
\end{figure}

\subsection*{解答}

\textbf{推理:}原来拉力$5N$静止不动,表示$f_{max静摩擦力}\ge 5N$。现在突然需要向右$8N$的力,弹簧不是瞬时力,只能提供$5N$,那么必定有$3N$水平向右的力,由静摩擦力提供。此时物体A:$f_{摩擦力} <  f_{max摩擦力}$不会动。
\begin{enumerate}
	
	
	\item 		\textbf{初始状态分析:}
	由图可知,弹簧连接在物体\( A \)的右侧和小车的右壁之间,且弹簧处于“被拉伸”状态,因此弹簧对物体\( A \)施加向右的拉力 \( F_{\text{弹}} = 5 \text{ N} \)。
	
	物体\( A \)处于静止状态,受力平衡。在水平方向上:
	\begin{align*}
		&\because F_{\text{弹}} + f = 0 \\
		&\therefore f = -F_{\text{弹}} = -5 \text{ N}
	\end{align*}
	即初始摩擦力大小为 5 N,方向水平向左。
	
	同时可知,最大静摩擦力 \( f_{\max} \) 至少能提供 5 N 的力,即 \( f_{\max} \geq 5 \text{ N} \)。
	
	\item 		\textbf{加速状态分析:}
	小车以 \( a = 0.8 \text{ m/s}^2 \) 向右加速。假设物体\( A \)相对于小车静止(即随车一起加速),则物体\( A \)的加速度也是 \( 0.8 \text{ m/s}^2 \)(向右)。
	
	根据牛顿第二定律,物体\( A \)所受合外力应为:
	\[ F_{\text{合}} = ma = 10 \times 0.8 = 8 \text{ N} \text{(向右)} \]
	
	此时,水平方向受力仍为弹簧拉力和摩擦力。根据假设相对静止,弹簧长度未变,拉力仍为 \( 5 \text{ N} \)(向右)。
	设此时的摩擦力为 \( f' \)(向右为正):
	\[ F_{\text{弹}} + f' = F_{\text{合}} \]
	\[ 5 + f' = 8 \]
	\[ f' = 3 \text{ N} \]
	
	结果表明,要使物体随车一起以 \( 0.8 \text{ m/s}^2 \) 加速,需要 3 N 向右的静摩擦力。
	
	\item 		\textbf{检验假设:}
	我们算出需要的静摩擦力为 3 N。而我们已知最大静摩擦力 \( f_{\max} \geq 5 \text{ N} \)。
	因为 \( 3 \text{ N} < 5 \text{ N} \),所以静摩擦力足以提供所需的加速度,物体\( A \) \textbf{不会} 发生相对滑动,确实和小车保持相对静止。
	
	\item 		\textbf{对比前后状态:}
	\begin{itemize}
		\item \textbf{弹簧拉力:} 始终为 5 N(因为没有相对滑动)。
		\item \textbf{摩擦力:} 由 5 N(向左)变为 3 N(向右)。
		\item \textbf{摩擦力大小:} 由 5 N 减小为 3 N。
	\end{itemize}
	
	\item 		\textbf{选项分析:}
	\begin{itemize}
		\item \textbf{A. 物体\( A \)可能会相对小车向右运动}:
		\cuo 我们已经验证物体相对于小车静止。即使发生滑动,物体相对于小车的加速度小于车的加速度,表现为相对后退(向左),绝不可能相对向右冲去。
		\item \textbf{B. 物体\( A \)受到的弹簧的拉力可能增大}:
		\cuo 物体相对静止,弹簧形变量不变,拉力不变。
		\item \textbf{C. 物体\( A \)受到的摩擦力大小可能不变}:
		\cuo。摩擦力大小由 5 N 变为 3 N,一定变了。
		\item \textbf{D. 物体\( A \)受到的摩擦力一定减小}:
		\dui 大小从 5 N 变为 3 N,确实减小了。
	\end{itemize}
\end{enumerate}
\textbf{答案:D}

如图\ref{fig3.2}所示,质量分别为 $m_A$ 和 $m_B$ 的 A、B 两物体沿着固定光滑斜面匀加速下滑,斜面倾角为 $\theta=30^\circ$,A 的上表面水平,且 A、B 始终保持相对静止,重力加速度为 $g$。则物体 A、B 的加速度大小为:\underline{$g\sin\theta$};B 受到 A 的摩擦力大小为:\underline{$M_B\sin\theta\cos\theta=\dfrac{\sqrt{3}}{4}M_B\cdot g$}、B 受到 A的支持力大小为:\underline{$M_Bg-N=M\cdot a\sin\theta\iff N=\dfrac{3}{4}m_Bg$}。
\begin{figure}[H]
	\centering
	\includegraphics[width=5cm]{pics/2.png}
	\caption{物体匀加速下滑示意图}
	\label{fig3.2}
\end{figure}
加速度有竖直向下分量,物体失重。

(多选)如图所示,物体A的质量为$M$,水平面光滑,不计滑轮的质量及摩擦。当在轻绳的$B$端挂一质量为$m$的物体时,物体$A$的加速度为$a_1$;当在轻绳的$B$端施以$F=mg$的竖直向下的拉力作用时,$A$的加速度为$a_2$,则()

\begin{enumerate}
	\item[A.] $a_1=\dfrac{m}{M}g$
	\item[B.] $a_2=\dfrac{m}{M}g$
	\item[C.] $a_1=a_2$
	\item[D.] $a_1<a_2$
\end{enumerate}

\begin{tikzpicture}[scale=0.8]
	% 绘制水平面和竖直墙面
	\draw[thick] (0,0) -- (6,0);
	\draw[thick] (6,0) -- (6,-6);
	\draw[pattern=north east lines] (0,-0) rectangle (6,-0.2);
	\draw[pattern=north east lines] (6,0) rectangle (5.8,-6);
	
	% 绘制物体A
	\draw[thick] (1,0) rectangle (2,1) node[midway] {A};
	
	% 绘制滑轮
	\filldraw[black] (6.15,0.15) circle (0.2);
	\draw[thick] (5.8,-0.2) arc (270:90:0.2);
	
	% 绘制绳子
	\draw[thick] (2,0.5) -- (6.3,0.5)  -- (6.3,-2) node[midway, right] {B};
	
	% 绘制拉力F
	\draw[thick, ->] (6.3,-2) -- (6.3,-4) node[below] {$F$};
\end{tikzpicture}